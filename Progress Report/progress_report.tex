\documentclass{article}
\usepackage{hyperref}

\title{Project Proposal: Using Computer Vision for Classification of Art}
\date{\today}

\author{Mike Micatka, Ari Brown, and Andrew Gleeman}

\begin{document}
\maketitle

\section{Problem}
The topic of art identification and classification is an ever-present issue in
the world of art. Experts are paid thousands of dollars to give completely
subjective opinions about the origins and qualities of paintings. We wish to
use computer vision and other analytical techniques to provide a more
quantitative analysis of the features of a painting. Using this quantification
of the features, we will attempt to classify the given painting and match it
to others like it. These similar paintings may be paintings by the same artist,
or just paintings in the same style (impressionist, realist, surrealist, and
so on). In doing so, we hope to create a tool for the automatic classification
of art.

\section{Technical Part}
Our implementation of this tool includes two largely distinct parts after
generating a database of about 500 images, roughly 100 images for the 5
artists we are examining. First, we will produce a number of algorithms
to extract quantitative data about the features of the paintings. We are
using the following tests for extracting quantitative data: color palette,
to determine the brightness and average color used in the paintings;
blob detection, to determine the number of focal features and the average
size of them; SIFT, to discern local features; Histogram of Oriented Gradients,
which discerns local object appearance from the gradient of intensities;
local binary patterns, to synergize with HoG; corner detection, for the
jaggedness of patterns; edge detection, for a similar manner; and texton
histograms, which finds texture patterns. \\

The second part of our project involves using this data to sort and classify
the images. We plan on using support vector machines (SVM) to facilitate the
training and educated guessing of art pieces. Support vector machines simply
determine, given two classes of points and data for each class, to which class
a new point would belong to. \\

We will run the experiment by training our support vector machine and
then testing it on naive images. We will be comparing outcome between the
actual artist of the painting and another artist in our database. \\

Error will be determined by the correct percentage of artists guessed. This
will be compared to the baseline of Blessing and Wen's work. Blessing and Wen
managed to succeed over 90\% of the time with all of the artists that they
chose. We will consider over 80\% to be a success, and over 95\% to be an
improvement.

\section{Milestones Completed}
The following is a list of milestones and sub-goals that we have achieved in
order to realize this project. It also includes rough expectations for the
future tasks for when we hope to have them completed. \\ 

\begin{itemize}
  \item October 30, 2012: Collect, organize, and construct a database of images.
  We have compiled roughly 500 images for 5 artists. We also have a clean
  interface with the database that makes creating and editing the information
  for the images simple.

  \item November 8, 2012: Complete implementations of various feature extraction
  algorithms
  We have the implementations completed for 5 of the 8, and we are planning on
  getting the remaining 3 done by November 28, 2012. We missed our original
  goal of November 8, but have not lost too much ground.

  \item November 25, 2012: Complete progress report
  This is very meta and is happening at this very moment.
\end{itemize}

\section{Milestones Yet Incomplete}
\begin{itemize}
  \item December 1, 2012: Implement various machine learning classification
  algorithms
  \item December 3, 2012: Test, evaluate, and refine algorithm
  \item December 4, 2012: Present
  \item December 17, 2012: Finish implementation
\end{itemize}

\section{Division of Labor}
We plan on collaborating on most aspects of the project, with smaller tasks
assigned as appropriate. For instance, we may individually implement special
feature extraction algorithms or a particular machine learning approach. \\

\appendix
\section{References}
We identified a Stanford University paper entitled "Using Machine Learning for
Identification of Art Paintings" (Blessing, Wen, 2010). This paper describes a
project comparable to our intended project, with a few key differences. The
authors implemented a similarly structured program with a heavier and more
technical machine learning focus. We plan to focus the majority of our
attention on the feature extraction tools, as they are the ones that are most
relevant to the class. A link to the paper can be found
\href{http://cs229.stanford.edu/proj2010/BlessingWen-UsingMachineLearningForIdentificationOfArtPaintings.pdf}{here}.

\end{document}

